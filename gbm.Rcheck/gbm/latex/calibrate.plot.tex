\HeaderA{calibrate.plot}{Calibration plot}{calibrate.plot}
\keyword{hplot}{calibrate.plot}
\begin{Description}\relax
An experimental diagnostic tool that plots the fitted values versus the actual average values.
Currently developed for only \code{distribution="bernoulli"}.
\end{Description}
\begin{Usage}
\begin{verbatim}
calibrate.plot(y,p,
               distribution="bernoulli",
               replace=TRUE,
               line.par=list(col="black"),
               shade.col="lightyellow",
               shade.density=NULL,
               rug.par=list(side=1),
               xlab="Predicted value",
               ylab="Observed average",
               xlim=NULL,ylim=NULL,
               knots=NULL,df=6,
               ...)
\end{verbatim}
\end{Usage}
\begin{Arguments}
\begin{ldescription}
\item[\code{y}] the outcome 0-1 variable 
\item[\code{p}] the predictions estimating E(y|x) 
\item[\code{distribution}] the loss function used in creating \code{p}.
\code{bernoulli} and \code{poisson} are currently the
only special options. All others default to squared error
assuming \code{gaussian}
\item[\code{replace}] determines whether this plot will replace or overlay the current plot.
\code{replace=FALSE} is useful for comparing the calibration of several
methods
\item[\code{line.par}] graphics parameters for the line 
\item[\code{shade.col}] color for shading the 2 SE region. \code{shade.col=NA} implies no 2 SE
region
\item[\code{shade.density}] the \code{density} parameter for \code{\LinkA{polygon}{polygon}}
\item[\code{rug.par}] graphics parameters passed to \code{\LinkA{rug}{rug}}
\item[\code{xlab}] x-axis label corresponding to the predicted values
\item[\code{ylab}] y-axis label corresponding to the observed average
\item[\code{xlim,ylim}] x and y-axis limits. If not specified te function will select
limits
\item[\code{knots,df}] these parameters are passed directly to 
\code{\LinkA{ns}{ns}} for constructing a natural spline 
smoother for the calibration curve
\item[\code{...}] other graphics parameters passed on to the plot function 
\end{ldescription}
\end{Arguments}
\begin{Details}\relax
Uses natural splines to estimate E(y|p). Well-calibrated predictions
imply that E(y|p) = p. The plot also includes a pointwise 95
band.
\end{Details}
\begin{Value}
\code{calibrate.plot} returns no values.
\end{Value}
\begin{Author}\relax
Greg Ridgeway \email{gregr@rand.org}
\end{Author}
\begin{References}\relax
J.F. Yates (1982). "External correspondence: decomposition of the mean
probability score," Organisational Behaviour and Human Performance 30:132-156.

D.J. Spiegelhalter (1986). "Probabilistic Prediction in Patient Management
and Clinical Trials," Statistics in Medicine 5:421-433.
\end{References}
\begin{Examples}
\begin{ExampleCode}
library(rpart)
data(kyphosis)
y <- as.numeric(kyphosis$Kyphosis)-1
x <- kyphosis$Age
glm1 <- glm(y~poly(x,2),family=binomial)
p <- predict(glm1,type="response")
calibrate.plot(y, p, xlim=c(0,0.6), ylim=c(0,0.6))
\end{ExampleCode}
\end{Examples}

