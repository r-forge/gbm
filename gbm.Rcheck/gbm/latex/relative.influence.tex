\HeaderA{relative.influence}{Methods for estimating relative influence}{relative.influence}
\aliasA{gbm.loss}{relative.influence}{gbm.loss}
\aliasA{permutation.test.gbm}{relative.influence}{permutation.test.gbm}
\keyword{hplot}{relative.influence}
\begin{Description}\relax
Helper functions for computing the relative influence of each variable in the gbm object.
\end{Description}
\begin{Usage}
\begin{verbatim}
relative.influence(object, n.trees, scale., sort.)
permutation.test.gbm(object, n.trees)
gbm.loss(y,f,w,offset,dist,baseline)
\end{verbatim}
\end{Usage}
\begin{Arguments}
\begin{ldescription}
\item[\code{object}] a \code{gbm} object created from an initial call to \code{\LinkA{gbm}{gbm}}.
\item[\code{n.trees}] the number of trees to use for computations. If not provided, the
the function will guess: if a test set was used in fitting, the number of
trees resulting in lowest test set error will be used; otherwise, if
cross-validation was performed, the number of trees resulting in lowest
cross-validation error will be used; otherwise, all trees will be used.
\item[\code{scale.}] whether or not the result should be scaled. Defaults to \code{FALSE}.
\item[\code{sort.}] whether or not the results should be (reverse) sorted.
Defaults to \code{FALSE}.
\item[\code{y,f,w,offset,dist,baseline}] For \code{gbm.loss}: These components are the
outcome, predicted value, observation weight, offset, distribution, and comparison
loss function, respectively.
\end{ldescription}
\end{Arguments}
\begin{Details}\relax
This is not intended for end-user use. These functions offer the different
methods for computing the relative influence in \code{\LinkA{summary.gbm}{summary.gbm}}.
\code{gbm.loss} is a helper function for \code{permutation.test.gbm}.
\end{Details}
\begin{Value}
By default, returns an unprocessed vector of estimated relative influences.
If the \code{scale.} and \code{sort.} arguments are used, returns a processed
version of the same.
\end{Value}
\begin{Author}\relax
Greg Ridgeway \email{gregr@rand.org}
\end{Author}
\begin{References}\relax
J.H. Friedman (2001). "Greedy Function Approximation: A Gradient Boosting
Machine," Annals of Statistics 29(5):1189-1232.

L. Breiman (2001). "Random Forests," Available at \url{ftp://ftp.stat.berkeley.edu/pub/users/breiman/randomforest2001.pdf}.
\end{References}
\begin{SeeAlso}\relax
\code{\LinkA{summary.gbm}{summary.gbm}}
\end{SeeAlso}

