\HeaderA{pretty.gbm.tree}{Print gbm tree components}{pretty.gbm.tree}
\keyword{print}{pretty.gbm.tree}
\begin{Description}\relax
\code{gbm} stores the collection of trees used to construct the model in a 
compact matrix structure. This function extracts the information from a single
tree and displays it in a slightly more readable form. This function is mostly
for debugging purposes and to satisfy some users' curiosity.
\end{Description}
\begin{Usage}
\begin{verbatim}
pretty.gbm.tree(object, 
                i.tree = 1)
\end{verbatim}
\end{Usage}
\begin{Arguments}
\begin{ldescription}
\item[\code{object}] a \code{\LinkA{gbm.object}{gbm.object}} initially fit using \code{\LinkA{gbm}{gbm}}
\item[\code{i.tree}] the index of the tree component to extract from \code{object} 
and display 
\end{ldescription}
\end{Arguments}
\begin{Value}
\code{pretty.gbm.tree} returns a data frame. Each row corresponds to a node in 
the tree. Columns indicate
\begin{ldescription}
\item[\code{SplitVar}] index of which variable is used to split. -1 indicates a 
terminal node.
\item[\code{SplitCodePred}] if the split variable is continuous then this component
is the split point. If the split variable is categorical then this component
contains the index of \code{object\$c.split} that describes the categorical
split. If the node is a terminal node then this is the prediction.
\item[\code{LeftNode}] the index of the row corresponding to the left node.
\item[\code{RightNode}] the index of the row corresponding to the right node.
\item[\code{ErrorReduction}] the reduction in the loss function as a result of 
splitting this node.
\item[\code{Weight}] the total weight of observations in the node. If weights are all
equal to 1 then this is the number of observations in the node.
\end{ldescription}
\end{Value}
\begin{Author}\relax
Greg Ridgeway \email{gregr@rand.org}
\end{Author}
\begin{SeeAlso}\relax
\code{\LinkA{gbm}{gbm}}, \code{\LinkA{gbm.object}{gbm.object}}
\end{SeeAlso}

