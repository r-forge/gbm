\HeaderA{print.gbm}{~~function to do ... ~~}{print.gbm}
\keyword{models}{print.gbm}
\keyword{nonlinear}{print.gbm}
\keyword{survival}{print.gbm}
\keyword{nonparametric}{print.gbm}
\keyword{tree}{print.gbm}
\begin{Description}\relax
Display basic information about a \code{gbm} object.
\end{Description}
\begin{Usage}
\begin{verbatim}
print.gbm(x, ...)
\end{verbatim}
\end{Usage}
\begin{Arguments}
\begin{ldescription}
\item[\code{x}] an object of class \code{gbm}. 
\item[\code{...}] arguments passed to \code{print.default}. 
\end{ldescription}
\end{Arguments}
\begin{Details}\relax
Prints some information about the model object. In particular, the call
to the model fitting function is given, and the type of loss function
that was used is given, as is the total number of iterations.

If cross-validation was performed, the 'best' number of trees as
estimated
by cross-validation error is
dispalyed. If a test set was used, the 'best' number
of trees as estimated by the test set error is displayed.

The number of available predictors, and the number of those having
non-zero influence on predictions is given (which might be interesting
in data mining applications).

If K-class, Bernoulli or adaboost classification was performed,
the confusion matrix and prediction accuracy are printed (objects
being allocated to the class with highest probability for K-class
and Bernoulli). These classifications are performed on the entire
training
data using the model with the 'best' number of trees as described
above, or the maximum number of trees if the 'best' can't be
computed.

If the 'distribution' was specified as Gaussian, Laplace, quantile,
bisquare or t-distribution, a summary of the residuals is displayed.
The residuals are for the training data with the model at the 'best' number of trees, as
described above, or the maximum number of trees if the 'best' can't
be computed.
\end{Details}
\begin{Author}\relax
Harry Southworth, Daniel Edwards
\end{Author}
\begin{SeeAlso}\relax
\code{\LinkA{gbm}{gbm}}
\end{SeeAlso}
\begin{Examples}
\begin{ExampleCode}
library( gbm )
data( iris )
iris.mod <- gbm( Species ~ ., distribution="kclass", data=iris,
                 n.trees=2000, shrinkage=.01, cv.folds=5 )
iris.mod
data( lung )
lung.mod <- gbm( Surv(time, status) ~ ., distribution="coxph", data=lung,
                 n.trees=2000, shrinkage=.01, cv.folds=5 )
lung.mod
\end{ExampleCode}
\end{Examples}

